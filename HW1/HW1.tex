%-- coding: UTF-8 --

\documentclass[12pt]{article}

\title{Homework 01}

\usepackage{indentfirst}
\usepackage{geometry}
\usepackage{amsmath}
\usepackage{amssymb}
\usepackage{amstext}
\usepackage{amsthm}
\usepackage[ruled]{algorithm2e}
%\usepackage{algorithmic}
\usepackage{algpseudocode}
\usepackage[UTF8]{ctex}
\usepackage{txfonts}
\usepackage{bm}
\geometry{top=2cm,left=2cm,right=2cm,bottom=2cm}
\date{\today}

\begin{document}
	\linespread{0.6} %%行间距
	\setlength{\parskip}{0.3\baselineskip}
	\begin{center}
		\LARGE{\textbf{Effectively Solving Linear Ridge Regression}}
		
		\small{Student ID:518021911150 \text{} \text{}  \text{} \text{} Class:F1803405 \text{} \text{} \text{} \text{} Name:周睿文}
	\end{center}
	\section*{Problem:}
	\begin{itemize}
		\item Numerical computation of matrix inversion of $Z^T Z$ is expensive
		\item Instead we could use singular value decomposition (SVD) to lower the computation cost:
		\[ Z = UDV^T \]
		where:
		\begin{itemize}
			\item $U = (u_1, u_2, \cdots, u_p)$ is an $n\times p$ orthogonal matrix
			\item $D = \text{diag}(d_1, d_2, \cdots, d_p)$ is a $p\times p$ diagonal matrix \\
			consisting of the singular values $d_1 \geq d_2 \geq \cdots \geq d_p \geq 0$
			\item $V^T = (v_1^T, v_2^T, \cdots, v_p^T)$ is a $p\times p$ orthogonal matrix
		\end{itemize}
		\item Proof:
		\begin{align*}
		\hat{\beta}_\lambda^{\text{ridge}} &= \left(Z^T Z + \lambda I_p\right)^{-1}Z^T y \\
		&= V \mathop{\text{diag}}\limits_j\left(\frac{d_j}{d_j^2 + \lambda}\right)U^T y
		\end{align*}
	\end{itemize}
	
	\section*{Proof:}
	From the objective of linear ridge regression, we have:
	\begin{align*}
		J\left(\beta\right) &= \left(y-Z\beta\right)^T\left(y-Z\beta\right) + \lambda \beta^T\beta
	\end{align*}
	To minimize the cost function $J\left(\beta\right)$, we need:
	\begin{align*}
		\frac{\partial J}{\partial \beta} &= -2Z^T\left(y-Z\beta\right) + 2\lambda\beta = 0
	\end{align*}
	Hence we get:
	\begin{align*}
		Z^T y &= Z^T Z\beta + \lambda\beta \\
		\beta &= \left(Z^T Z + \lambda I_p\right)^{-1}Z^T y
	\end{align*}
	As will be shown later, the $p \times p$ matrix $\left(Z^T Z + \lambda I_p\right)$ is invertible as long as $\lambda > 0$.
	
	According to the definition of singular value decomposition, we let $Z = UDV^T$, where:
	\begin{itemize}
		\item $U = (u_1, u_2, \cdots, u_p)$ is an $n\times p$ orthogonal matrix
		\item $D = \text{diag}(d_1, d_2, \cdots, d_p)$ is a $p\times p$ diagonal matrix \\
		consisting of the singular values $d_1 \geq d_2 \geq \cdots \geq d_p \geq 0$
		\item $V^T = (v_1^T, v_2^T, \cdots, v_p^T)$ is a $p\times p$ orthogonal matrix
	\end{itemize}
	
	Therefore, we obtain:
	\begin{align*}
		Z^T Z &= \left(UDV^T\right)^T\left(UDV^T\right) \\
		&= VD^T U^T UDV^T \\
		&= VD^T DV^T \\
		&= VD^2 V^T
	\end{align*}
	
	As $V$ is an orthogonal matrix, we have $V^T V = I_p$, so we can get:
	\begin{align*}
		Z^T Z + \lambda I_p &= VD^2 V^T + \lambda I_p \\
		&= V\left(D^2 + \lambda I_p\right)V^T \\
		&= V \mathop{\text{diag}}\limits_j\left(d_j^2 + \lambda\right) V^T
	\end{align*}
	
	As long as $\lambda > 0$, we have $d_j^2 + \lambda > 0, \forall j\in\lbrace 0, 1, \cdots,p\rbrace$, so we can get $\left(Z^T Z + \lambda I_p\right)^{-1}$ as following:
	\begin{align*}
		V\mathop{\text{diag}}\limits_j\left(\frac{1}{d_j^2 + \lambda}\right)V^T \left(Z^T Z + \lambda I_p\right) &= 
		V\mathop{\text{diag}}\limits_j\left(\frac{1}{d_j^2 + \lambda}\right)V^T V \mathop{\text{diag}}\limits_j\left(d_j^2 + \lambda\right) V^T \\
		&= V\mathop{\text{diag}}\limits_j\left(\frac{1}{d_j^2 + \lambda}\right)\mathop{\text{diag}}\limits_j\left(d_j^2 + \lambda\right) V^T \\
		&= V V^T = I_p \\
		\Longrightarrow \left(Z^T Z + \lambda I_p\right)^{-1} &=
		V\mathop{\text{diag}}\limits_j\left(\frac{1}{d_j^2 + \lambda}\right)V^T
	\end{align*}
	
	Therefore, we reach the final result:
	\begin{align*}
		\beta &= \left(Z^T Z + \lambda I_p\right)^{-1}Z^T y \\
		&= V\mathop{\text{diag}}\limits_j\left(\frac{1}{d_j^2 + \lambda}\right)V^T \left(VD^T U^T\right) y \\
		&= V\mathop{\text{diag}}\limits_j\left(\frac{1}{d_j^2 + \lambda}\right)\mathop{\text{diag}}\limits_j\left(d_j\right) U^T y \\
		&= V \mathop{\text{diag}}\limits_j\left(\frac{d_j}{d_j^2 + \lambda}\right)U^T y
	\end{align*}
	\begin{flushright}
	$\blacksquare$
	\end{flushright}
\end{document}
